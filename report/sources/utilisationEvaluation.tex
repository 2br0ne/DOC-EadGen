

\subsection{Utilisation d'EadGen}

\paragraph{}Avec EadGen, un auteur peut préparer un document source balisé facilement, avec un simple éditeur de text. Le document est alors uploadé sur le site de EadGen et est associé à un langage disponible selon les préférences de l'auteur. Une foi la transformation vers effectuée, le site est présenté sous forme d'une archive contenant les documents \texttt{HTML}, les scripts, constituants de la charte graphique. 

L'ensemble est prêt à être déployé sur le site de l'auteur.

Il faut quelques minutes pour le serveur EadGen pour produire un site de quelques centaines de pages \texttt{HTML}.





\section{Évaluation}


\paragraph{}A sa sortie, en 2001, EadGen a apporté beaucoup à la pédagogie en ligne. L'un de ses points fort réside dans la séparation entre le contenu et la forme, permettant une personnalisation ainsi qu'un ré-emploi facilité et accru.


Le langage de balisage est léger et allège considérablement le codage de l'auteur, il permet, grâce aux personnalisations, de ne pas brider sa créativité et son initiative.


\paragraph{}Le fait que EadGen utilise un langage ouvert facilite grandement son adaptation aux différentes disciplines et à leurs spécificités. De plus il est bien adapté à l'utilisation de méta-données pour caractériser des éléments du cours. 


\paragraph{}Il ne faut pas nonplus oublier que la production de fichiers \texttt{HTML} n'est qu'un cas particulier des possibilités de EadGen, il est en effet aussi facile, avec \texttt{XSL-T}, de produire du code d'accès à une base de données pour stocker ou extraire des éléments en fonction, par exemple des méta-données.