\subsubsection{Du document en langage auteur au document XML}

Chaque document est donc écrit dans
un langage spécifique à l'environnement Eadgen. Il est ainsi
composé de balises formant un arbre. Le responsable de collection
a définit dans un fichier XML les balises utilisées
dans le langage auteur. Les traitements sémantiques et les aspects
plus techniques sont ainsi réalisés par le responsable de collection
et l'auteur du document n'a ainsi aucune difficulté à écrire son
document.

Le traducteur Eadgen va s'employer à transformer ces documents
écrits en langage auteur en document XML, bien formés au sens de
XML et conforme à la DTD formée par le fichier XML créé par le responsable
de collection. La structure du document XML ainsi formée correspond
donc exactement à la structure de l'arbre définit par l'auteur.

\subsubsection{Du document XML au document HTML}

Eadgen a aussi pour objectif de convertir les données des documents
XML en document HTML lisible, par exemple sur le web. La production de
ces documents HTML se fait via l'utilisation de XSL-T. C'est grace à
ce dernier que nous allons pouvoir transformer des sous-arbres de documents
XML en sous-arbres HTML.

Léger inconvénient néanmoins : l'écriture de document XSL-T est très lourde.
C'est pourquoi Eadgen est équipé d'un second pré-processeur qui a pour role
de transformer en XSL-template les maquettes que le graphiste fournit au
format HTML. Une convention a été mise en place pour que ce dernier puisse
indiquer, via les commentaires, que tel modèle sert pour telle balise.

C'est une syntaxe simple et satisfaisante pour répondre à cette problématique.
