Avec l'essor de l'Internet, et des techniques permettant de mettre
en ligne des ressources pédagogiques, se pose la question du besoin
continuel de formation et de parcours pédagogiques \og sur mesure \fg.

Ceci créant un besoin de transformer des ressources pédagogiques sous
une ancienne forme en documents numériques formatés. Pour cela, il
existe diverses techniques, mais la plupart requiert un certain
nombre de bagages de compétences que tout les auteurs de ressources
pédagogiques ne possèdent pas obligatoirement. 

Des scientifiques du CNRS, à l'occasion du concours du LIRMM, ont mis
au point un environnement auteur, appelé Eadgen. Celui-ci a pour but
de simplifier au maximum le travail des auteurs de ressources pédagogiques.
Le fonctionnement de cet outil repose sur le langage XML mais permet
de simplifier la tache de tout les acteurs entrant en jeu dans la 
création de telles ressources pédagogiques.