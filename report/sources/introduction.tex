Avec l'essor de l'Internet et des techniques permettant de mettre
en ligne des ressources pédagogiques, se pose la question du besoin
continuel de formations et de parcours pédagogiques \og sur mesure \fg.

Ceci créant un besoin de transformer des ressources pédagogiques d'une ancienne forme en documents numériques formatés. Pour cela, il
existe diverses techniques, mais la plupart requièrent des compétences que tous les auteurs de ressources
pédagogiques ne possèdent pas obligatoirement. 

Des scientifiques du CNRS, avec le concours du LIRMM, ont mis
au point un environnement auteur, appelé Eadgen, qui repose sur le langage XML. Celui-ci a pour but
de simplifier au maximum le travail de tous les acteurs entrant en jeu dans la création de telles ressources pédagogiques.