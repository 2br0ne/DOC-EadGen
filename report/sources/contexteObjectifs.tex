EadGen est un environnement auteur, il s'agit d'un logiciel facilitant la production de documents pédagogiques en ligne. Il met à disposition chaîne de production entièrement paramétrable pour la création d'un document interactif à partir de "matière première pédagogique".

\subsection{Les documents produits}

\paragraph{}La production de documents pédagogiques en ligne ne peut se limiter aux simples publications des supports de cours car ceux-ci ne sont que des supports, des compléments, comme l'indique leur désignation. 

Il faut donc, pour obtenir des cours efficaces, mettre à disposition de l'apprenant tout ce dont il a besoin, c'est à dire aussi bien les exercices que des auto-évaluations. Il faut aussi ajouter toutes les information transmises oralement par l'enseignant, ou même ses gestes, les changements d'intonation de voix, les anecdotes. 

La médiatisation d'un document est la transformation de ces informations pour qu'elles apparaissent dans la production finale en ligne. 

\paragraph{}Toutes ces informations sont trop imprévisibles et changeantes d'un cours à l'autre qu'il est impossible d'automatiser et de généraliser un processus permettant la médiatisation de l'ensemble dans un format interactif convenable pour l'apprenant. 

Il est alors nécessaire d’allouer une grande partie du temps de production de la ressource pédagogique dans la conception, c'est à dire en utilisant des techniques de production automatisées mais paramétrables au maximum.

\subsection{Méthodes existantes}
\subsubsection{Production page par page}
\paragraph{}Il s'agit d'une méthode donnant de très bons résultats grâce à une médiatisation très complète des documents. Il s'agit d'une production spécifique à chaque leçon, permettant la plus grande fidélité avec des besoins de l'enseignant. 

\paragraph{}C'est, par contre une méthode extrêmement gourmande en temps afin d'obtenir un résultat particulier pour chaque page, un généralisation conduisant à la perte des subtilités d'une leçon par rapport à une autre. 

La maintenance est elle aussi particulièrement délicate, en effet un changement un minimum significatif entraîne quasiment la ré-élaboration de la page. 

De plus, et c'est le pire côté de cette technique, une évolution dans le modèle de navigation, la charte graphique ou autre élément général du site entraîne la mise à jour manuelle de la totalité des pages ainsi produites, donc un temps considérable.   

\subsubsection{Pages dynamiques}
\paragraph{}Cette technique repose sur l'utilisation de scripts par le serveur permettant de réaliser algorithmiquement les pages. Les données représentées sont alors fortement structurées et doivent répondre à un schéma global. 

Si cette forte structuration peut convenir à certaines disciplines scientifiques, elle est très difficilement généralisable à toutes les matières.

\paragraph{}L'avantage de cette technique concerne l'adaptabilité. Une foi les scripts de création des pages opérationnels, un ajout de page, une modification mineure ou majeure, la maintenance s'exécutent rapidement et simplement. 

Par contre la modification de ces scripts pour intégrer de nouvelles catégories de notions, par exemple est bien plus coûteuse car requière de réelles compétences de programmation.

\paragraph{}L'inconvénient de cette technologie réside dans sa généralisation à l'extrême, il est en effet extrêmement coûteux d'ajouter à ces scripts des fonctionnalités autres de la simple répétition d'actions dans un cadre défini. Il faut alors des spécialistes en informatique présentant aussi des compétences en communication. 

\subsubsection{Transformation de structure}
\paragraph{}Située entre les deux techniques précédentes, la transformation de structure repose sur l'utilisation du langage \texttt{XSL-T}. Ce langage sert à spécifier des transformation de structure entre un langage source: le \texttt{XML} et un langage cible, par exemple le \texttt{HTML}. 

Grâce à cet outil, la production des pages cibles avec l'utilisation de leur contenu et du modèle (\texttt{XSL-T}) est totalement automatisable.

Cette technique permet de conserver la flexibilité de la production page par page tout en exploitant l'économie des pages dynamiques. Il s'agit de la technique suggérée par W3C.

\paragraph{}Le problème de cette méthode est liée au langage de spécification du contenu: le \texttt{XML} qui est presque aussi lourd à l'écriture que le \texttt{HTML}. De même que la production des fichiers \texttt{XSL-T} demande presque autant de compétences que la production de scripts des pages dynamiques.

Une solution est donc de proposer un intermédiaire au \texttt{XML} pour l'utilisateur, en proposant un langage plus simple et léger. L'encapsulation du \texttt{XML} permet alors de continuer à tirer partit de la puissance du langage sans en subir les inconvénients. 
