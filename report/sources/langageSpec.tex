Comme expliqué précédemment, le responsable de la collection est en
charge de la liste des balises disponibles et de la sémantique
correspondante. Un langage de spécifications de ses balises doit
donc etre mis à disposition de l'utilisateur de l'environnement.

Il s'agit ainsi de construire le vocabulaire et la sémantique
du langage utilisé par l'auteur. Pour cela, Eadgen propose de délivrer
simplement un prototype de chaque balise dans lequel les trois
composantes de la balise sont définies, ainsi que les règles
de traitements pour la création du document XML. En dernier, le
nom du fichier XSL-template correspondant à la balise.

Voyons un exemple simple de définition de cette balise décrite dans un document XML par le responsable de collection :
\begin{verbatim}
	<Balise
		nom = «tox» priorité = «50»
		concerne = «nom du produit concerne»
		porte = «consignes de sécurité associées»
		detecte_occurrence= «bulle»
		cree_bulle_avec= «contenu»
		table_associée = «Produits Toxiques»
		references_croisée = «oui»
		nom_xml= «prodtox» />
\end{verbatim}
