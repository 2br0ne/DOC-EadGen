L'objectif principal du langage auteur dans Eadgen est de simplifier
la tache des utilisateurs de l'environnement. Il
est également nécessaire de concevoir un langage le plus extensible possible
pour couvrir un maximum de ressources pédagogiques.
Ainsi, tous les éléments du langage sont simplifiées au maximum.

Chaque balise doit se trouver en début de ligne et est précédée
du symbole \og \$ \fg, ceci la rendant plus lisible. Syntaxiquement, une
balise est constituée de trois éléments :
\begin{itemize}
	\item \emph{son nom} ;
	\item \emph{un champ de méta-données} optionnel ;
	\item \emph{un paramètre principal} optionnel.
\end{itemize}
La \emph{portée} d'une balise, c'est-à-dire les données sur lesquelles
elle influe, est définie par un système simple de priorités. Ce système
suit la règle : \emph{L'effet de la
balise perdure, jusqu'à rencontre d'une balise plus importante qu'elle}.

Une balise est donc de la forme : \\
\texttt{\$sonNom (ses méta-données) son paramètre principal}

Les méta-données d'une balise sont des informations qui n'apparaissent
pas dans le document cible mais qui sont néanmoins nécessaires. Le paramètre
principal peut correspondre à un titre ou une information supplémentaire.
